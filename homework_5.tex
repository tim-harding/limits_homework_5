\documentclass[12pt]{article}

\usepackage{fouriernc}
\usepackage{amssymb}
\usepackage{amsmath}
\usepackage{amsfonts}
\usepackage[utf8]{inputenc}
\usepackage[T1]{fontenc}
\usepackage[margin=1in]{geometry}
\usepackage{graphicx}

\graphicspath{ {./images/} }

\newcommand{\curly}[1]{\left\{      #1 \right\}     }
\newcommand{\round}[1]{\left(       #1 \right)      }
\newcommand{\hard} [1]{\left[       #1 \right]      }
\newcommand{\abs}  [1]{\left|       #1 \right|      }
\newcommand{\floor}[1]{\left\lfloor #1 \right\rfloor}
\newcommand{\ceil} [1]{\left\lceil  #1 \right\rceil }
\newcommand{\R}    [0]{\mathbb{R}                   }
\newcommand{\Z}    [0]{\mathbb{Z}                   }
\newcommand{\N}    [0]{\mathbb{N}                   }
\newcommand{\iif}  [0]{\Longleftrightarrow          }

\setlength{\parindent}{0in}

\title{Homework 5}
\author{Tim Harding}

\begin{document}
\maketitle

\section*{4.3.13}

\subsection*{2}
Testing the convergence of $\sum_{n=2}^\infty \frac{1}{n (\ln(n))^2}$
\begin{align*}
    u &= \ln(x) \\
    du &= \frac{dx}{x} \\
    dx &= x\ du
\end{align*}
\begin{align*}
     & \int_{\ln(2)}^\infty u^{-2}\ du \\
    =& \hard{-\frac{1}{u}}_{\ln(2)}^\infty \\
    =& -\frac{1}{\infty} - \round{-\frac{1}{\ln(2)}} \\
    =& \frac{1}{\ln(2)}
\end{align*}
The series converges.



\subsection*{3}
Testing the convergence of $\sum_{n=1}^\infty \frac{1}{n^2 + 1}$.
\begin{align*}
    x &= \tan(u) \\
    dx &= \sec^2(u)\ du \\
\end{align*}
\begin{align*}
    1 &= \tan(x) \\
    x &= \arctan(1) \\
    x &= \frac{\pi}{4}
\end{align*}
\begin{align*}
    \infty &= \tan(x) \\
    x &= \arctan(\infty) \\
    x &= \frac{\pi}{2}
\end{align*}
\begin{align*}
     & \int_\frac{\pi}{4}^\frac{\pi}{2} \frac{\sec^2(u)}{1 + \tan^2(u)} du \\
    =& \int_\frac{\pi}{4}^\frac{\pi}{2} du \\
    =& \hard{u}_{\frac{\pi}{4}}^\frac{\pi}{2} \\
    =& \frac{\pi}{2} - \frac{\pi}{4} \\
    =& \frac{\pi}{4}
\end{align*}
The series converges.



\subsection*{4}
Testing the convergence of $\sum_{n=1}^\infty \frac{3n^2 + 17n - 5}{n^5 + 48n^4 + 2007}$.

\begin{align*}
     & \lim_{n\to\infty} \frac{\frac{3n^2 + 17n - 5}{n^5 + 48n^4 + 2007}}{\frac{1}{n^3}} \\
    =& \lim_{n\to\infty} \frac{3n^5 + 17n^4 - 5n^3}{n^5 + 48n^4 + 2007} \\
    =& 3
\end{align*}
Since $\frac{1}{n^3}$ converges and $0<L<\infty$ we see that the series converges.



\subsection*{5}
Testing the convergence of $\sum_{n=2}^\infty \frac{2^n}{3^n + 4}$.
\begin{align*}
     & \lim_{n\to\infty} \frac{\frac{2^n}{3^n+4}}{\round{\frac{2}{3}}^n} \\
    =& \lim_{n\to\infty} \frac{2^n 3^n}{(3^n + 4)2^n} \\
    =& \lim_{n\to\infty} \frac{3^n}{3^n + 4} \\
    =& 1
\end{align*}
Since $\round{\frac{2}{3}}^n$ is convergent, our series is convergent.



\subsection*{8}
Testing the convergence of $\sum_{n=2}^\infty \frac{2 + \cos(n)}{n}$.

Since $\frac{2+\cos(n)}{n} \geq \frac{1}{n}$, we know that the series diverges.



\subsection*{10}
Testing the convergence of $\sum_{n=2}^\infty \frac{1}{\sqrt{n(n+1)(n+2)}}$.

\begin{align*}
     & \lim_{n\to\infty} \frac{\frac{1}{\sqrt{n(n+1)(n+2)}}}{\frac{1}{\sqrt{n^3}}} \\
    =& \sqrt{\lim_{n\to\infty} \frac{n^3}{n(n+1)(n+2)}} \\
    =& 1
\end{align*}
Since $\frac{1}{\sqrt{n^3}}$ converges, so too does the series.



\section*{4.3.17}

\subsection*{3}
We know that $\forall n \geq 1$, $\frac{a_n}{n} \leq a_n$, so the series converges.

\subsection*{4}
We know that $\sin(x) \leq x$ for all positive inputs, so the series converges.



\section*{4.3.A}
Determine the convergence or divergence of $\sum_{n=2}^\infty \frac{1}{n(\ln(n))^{k}}$ for $k>1$.
\begin{align*}
     & \int_{2}^\infty \frac{dx}{x (\ln(x))^k} \\
    =& \int_{\ln(2)}^\infty u^{-k}\ du \\
    =& \hard{\frac{1}{-k+1} u^{-k+1}}_{\ln(2)}^\infty \\
    =& \frac{1}{1-k} \round{\underbrace{\infty^{1-k}}_{1-k<0 \implies \infty^{1-k} \to 0} - (\ln(2))^{1-k}} \\
    =& -\frac{(\ln(2))^{1-k}}{1-k} < \infty
\end{align*}
The series converges.



\section*{4.3.B}
Write down detailed proof for Theorem 4.3.9 (a2):

$\forall n \in \N$, $a_n > 0$, $b_n > 0$, $\lim_{n\to\infty} \frac{a_n}{b_n} = 0$, $\abs{\sum_{n=1}^\infty b_n} < \infty$ $\implies \abs{\sum_{n=1}^\infty a_n} < \infty$

\bigskip\medskip

Let $B = \sum_{n=1}^\infty b_n$. Then $\forall n_0 \in \N$, $\sum_{n=n_0}^\infty b_n \leq B$. Suppose that $\exists N \in \N$, $\forall n \in \N : n > N \implies \abs{\frac{a_n}{b_n} - L} < 1$. This corresponds to the consequence of an $\epsilon - \delta$ proof of $\frac{a_n}{b_n}$ with $\epsilon = 1$. Then $\forall n \in \N : n \geq N+1 \implies a_n < (L+1)b_n$. When we choose $L=0$, we know that $a_n < b_n$. If we choose $n_0 = N+1$ and use the direct comparison result that $0 \leq a_n \leq b_n$, $\forall n \in \N : n > n_0$, $\abs{\sum_{n=n_0}^\infty b_n} < \infty$ $\implies \abs{\sum_{n=n_0}^\infty a_n} < \infty$ to see that $\sum_{n=N+1}^\infty a_n$ converges. We also know that $\sum_{n=1}^\infty a_n < \infty$ is a further consequence by using that fact that $\sum_{n=1}^\infty a_n < \infty \iif \sum_{n=n_0}^\infty a_n < \infty$.


\section*{4.4.10}

\subsection*{2}
Testing convergence of $\sum_{n=1}^\infty \frac{n^n}{n!}$.
\begin{align*}
     & \lim_{n\to\infty} \frac{\frac{(n+1)^{n+1}}{(n+1)!}}{\frac{n^n}{n!}} \\
    =& \lim_{n\to\infty} \frac{(n+1)^{n+1}}{n^n (n+1)} \\
    =& \lim_{n\to\infty} \frac{(n+1)^n}{n^n} \\
    =& \lim_{n\to\infty} \round{\frac{n+1}{n}}^n \\
    =& e > 1
\end{align*}
The series diverges.



\subsection*{3}
Testing convergence of $\sum_{n=1}^\infty \frac{(n!)^2}{(2n)!}$.
\begin{align*}
     & \lim_{n\to\infty} \frac{\frac{((n+1)!)^2}{(2(n+1))!}}{\frac{(n!)^2}{(2n)!}} \\
    =& \lim_{n\to\infty} \frac{((n+1)!)^2 (2n)!}{(2n+2)!\ (n!)^2} \\
    =& \lim_{n\to\infty} \round{\frac{(n+1)!}{n!}}^2 \frac{(2n)!}{(2n+2)!} \\
    =& \lim_{n\to\infty} \frac{(n+1)^2}{(2n+1)(2n+2)} \\
    =& \frac{1}{4} < 1
\end{align*}
The series converges.



\subsection*{5}
Testing convergence of $\sum_{n=1}^\infty \frac{n!}{(n+2)!}$.

\begin{align*}
     & \lim_{n\to\infty} \frac{\frac{(n+1)!}{(n+3)!}}{\frac{n!}{(n+2)!}} \\
    =& \lim_{n\to\infty} \frac{(n+1)! (n+2)!}{n! (n+3)!} \\
    =& \lim_{n\to\infty} \frac{n+1}{n+3} \\
    =& 1
\end{align*}
The ratio test is inconclusive.
\begin{align*}
     & \sum_{n=1}^\infty \frac{1}{(n+1)(n+2)} \\
    =& \sum_{n=1}^\infty \frac{1}{n^2 + 3n + 2}
\end{align*}
\begin{align*}
     & \lim_{n\to\infty} \frac{\frac{1}{n^2 + 3n + 2}}{\frac{1}{n^2}} \\
    =& \lim_{n\to\infty} \frac{n^2}{n^2 + 3n + 2} \\
    =& 1
\end{align*}
By the limit comparison test, the sum is convergent.



\subsection*{6}
Testing convergence of $\frac{2\sqrt{2}}{9801} \sum_{n=1}^\infty \frac{(4n)!\ (1103+26390n)}{(n!)^4 (396)^{4n}}$.
\begin{align*}
     & \lim_{n\to\infty} \frac{\frac{(4(n+1))!\ (1103+26390(n+1))}{((n+1)!)^4 (396)^{4(n+1)}}} {\frac{(4n)!\ (1103+26390n)}{(n!)^4 (396)^{4n}}} \\
    =& \lim_{n\to\infty} \frac{(4(n+1))!\ (1103+26390(n+1))}{(4n)!\ (1103+26390n)} \frac{(n!)^4 (396)^{4n}}{((n+1)!)^4 (396)^{4(n+1)}} \\
    =& \lim_{n\to\infty} \frac{(4n+1)(4n+2)(4n+3)(4n+4) (27493+26390n))}{(1103+26390n)} \round{\frac{n!\ (396)^n}{(n+1)!\ (396)^{n+1}}}^4 \\
    =& \lim_{n\to\infty} \frac{(4n+1)(4n+2)(4n+3)(4n+4) (27493+26390n))}{(1103+26390n)} \round{\frac{1}{(n+1) 396}}^4 \\
    =& \frac{4^4 \times 26390}{26390 \times 396^4} \\
    =& \round{\frac{4}{396}}^4 < 1
\end{align*}
The series converges.



\section*{4.4.12}
Given
\begin{align*}
    a_n = \begin{cases}
        \frac{n^2}{e^n} & \exists k \in \Z : n = 2k+1 \\
        \frac{2007}{e^n} & \exists k \in \Z : n = 2k
    \end{cases}
\end{align*}
show that $\sum_{i=1}^\infty a_n$ is convergent.

\begin{align*}
     & \sum_{i=1}^\infty \round{\frac{(2i-1)^2}{e^{2i-1}} + \frac{2007}{e^{2i}}} \\
    =& e \sum_{i=1}^\infty \frac{(2i-1)^2}{e^{2i}} + 2007 \sum_{i=1}^\infty e^{-2i} \\
    =& e \sum_{i=1}^\infty \frac{4i^2 - 4i + 1}{e^{2i}} + 2007 \sum_{i=1}^\infty e^{-2i} \\
    =& 4e \sum_{i=1}^\infty i^2 e^{-2i} - 4e \sum_{i=1}^\infty ie^{-2i} + e \sum_{i=1}^\infty e^{-2i} + 2007 \sum_{i=1}^\infty e^{-2i} \\
\end{align*}

\begin{align*}
     & \lim_{i\to\infty} \frac{\frac{(i+1)^2}{e^{2(i+1)}}} {\frac{i^2}{e^{2i}}} \\
    =& \lim_{i\to\infty} \frac{(i+1)^2}{i^2} \frac{e^{2i}}{e^{2(i+1)}} \\
    =& \lim_{i\to\infty} \round{\frac{i+1}{i} \frac{e^{i}}{e^{i+1}}}^2 \\
    =& \lim_{i\to\infty} \round{\round{1+\frac{1}{i}} \frac{1}{e}}^2 \\
    =& \round{ \frac{1}{e} \round{1 + \lim_{i\to\infty} \frac{1}{i}} }^2 \\
    =& \frac{1}{e^2} < 1
\end{align*}

$\sum_{i=1}^\infty i^2 e^{-2i}$ is convergent. Since $i e^{-2i} < i^2 e^{-2i}$ and $e^{-2i} < i^2 e^{-2i}$, these series converge as well. Therefore, the entire series converges.



\section*{4.4.13}
For $n \geq 3$, $\frac{5n+1}{4n+3} > 1$. Therefore, each term is increasing over time. For the series to be convergent, the terms should approach 0. Instead, they are growing. We know that the series diverges.



\section*{4.4.15}
To begin, we let $L = \lim_{n\to\infty} \abs{a_n}^\frac{1}{n}$. We have two cases:
\begin{enumerate}
    \item When $L<1$, we need to show that the series is absolutely convergent and hence convergent. There exists $r \in (L, 1)$ and $N \in \N$ such that $\forall n \geq N$ we have
    \begin{align*}
        \abs{a_n}^\frac{1}{n} &< r \\
        \abs{a_n} &< r^n
    \end{align*}
    Because $\sum_{n=N}^\infty \abs{a_n}$ is less than a geometric series we know that it converges. Next, we write the full series
    \begin{align*}
        \sum_{n=1}^\infty \abs{a_n} = \sum_{n=1}^{N-1} \abs{a_n} + \sum_{n=N}^\infty \abs{a_n}
    \end{align*}
    We have written the series as the sum of a finite sum of finite terms and a convergent infinite series. Therefore, the series converges.
    \item When $L>1$, we know that $\exists N \in \N$ such that $\forall n \geq N$ we have
    \begin{align*}
        \abs{a_n}^\frac{1}{n} &> 1 \\
        \abs{a_n} &> 1
    \end{align*}
    We know that if the terms of a series do not approach zero in the limit the series diverges. Therefore, we know that $\sum_{n=1}^\infty a_n$ diverges.
\end{enumerate}



\section*{4.4.16}
We need a good test for $\frac{\sqrt[n]{e}}{n^2}$. The root test is not a good fit here. The terms don't lend themselves well to $u$ substitution or integration by parts either, so the integral test doesn't work well either. Limit comparison is difficult as well, since dividing by the convergent $\frac{1}{n^2}$ yields a divergent seqence and dividing by the divergent $\sqrt[n]{e}$ yields a convergent sequence, so the tests are inconclusive. The radical on $e$ makes the ratio test tricky as well, and it doesn't quite work out:
\begin{align*}
     & \lim_{n\to\infty} \frac{\frac{e^\frac{1}{n+1}}{(n+1)^2}} {\frac{e^\frac{1}{n}}{n^2}} \\
    =& \lim_{n\to\infty} \frac{e^\frac{1}{n+1}}{e^\frac{1}{n}} \frac{n^2}{(n+1)^2} \\
    =& \lim_{n\to\infty} e^{\frac{1}{n+1} - \frac{1}{n}} \frac{n^2}{(n+1)^2} \\
    =& \lim_{n\to\infty} e^{\frac{n-(n+1)}{n(n+1)}} \frac{n^2}{(n+1)^2} \\
    =& \lim_{n\to\infty} e^{\frac{1}{n(n+1)}} \frac{n^2}{(n+1)^2} \\
    =& \lim_{n\to\infty} e^{\frac{1}{n(n+1)}} \round{ \lim_{n\to\infty} \frac{n}{n+1}}^2 \\
    =& 1 \times 1^2 = 1
\end{align*}
This is inconclusive. I have exhausted my options and I still don't know how to prove the convergence of this sequence. Sadness.



\section*{5.1.12}

\subsection*{2}
\subsubsection*{Problem}
\begin{align*}
    \sum_{n=1}^\infty \frac{x^n}{n2^n}
\end{align*}

\subsubsection*{Solution}
\begin{align*}
     & \lim_{n\to\infty} \frac{\frac{x^{n+1}}{(n+1)2^{n+1}}}{\frac{x^n}{n2^n}} \\
    =& \lim_{n\to\infty} \frac{x^{n+1}}{x^n} \frac{n2^n}{(n+1)2^{n+1}} \\
    =& \frac{x}{2} \lim_{n\to\infty} \frac{n}{n+1} \\
    =& \frac{x}{2}
\end{align*}
\begin{align*}
    -1 < &\frac{x}{2} < 1 \\
    -2 < &x < 2 \\
\end{align*}
The radius of convergence is $2$ and the interval of convergence is $(-2, 2)$.



\subsection*{3}
\subsubsection*{Problem}
\begin{align*}
    \sum_{n=1}^\infty \frac{(x-1)^n}{\ln(n+2)}
\end{align*}

\subsubsection*{Solution}
\begin{align*}
    =& \lim_{n\to\infty} \frac{\frac{(x-1)^{n+1}}{\ln((n+1)+2)}}{\frac{(x-1)^n}{\ln(n+2)}} \\
    =& \lim_{n\to\infty} \frac{(x-1)^{n+1}}{(x-1)^n} \frac{\ln(n+2)}{\ln(n+3)} \\
    =& (x-1) \ln\round{\lim_{n\to\infty} \frac{n+2}{n+3}} \\
    =& (x-1) \ln(0) \\
    =& x-1
\end{align*}
\begin{align*}
    -1 &< x-1 < 1 \\
    0 &< x < 2 \\
\end{align*}
The radius of convergence is $1$ and the interval of convergence is $(0,2)$.



\subsection*{4}
\subsubsection*{Problem}
\begin{align*}
    \sum_{n=1}^\infty \round{1+\frac{1}{n}}^n (x+e)^n
\end{align*}

\subsubsection*{Solution}
\begin{align*}
     & \lim_{n\to\infty} \sqrt[n]{\round{1+\frac{1}{n}}^n (x+e)^n} \\
    =& (x+e) \lim_{n\to\infty} \round{1+\frac{1}{n}} \\
    =& x+e
\end{align*}
\begin{align*}
    -1 &< x+e < 1 \\
    -1-e &< x < 1-e \\
\end{align*}
The radius of convergence is $1$ and the interval of convergence is $(-1-e, 1-e)$.



\section*{5.1.15}
\begin{align*}
    f(x) &= \sum_{n=0}^\infty \frac{x^n}{n!} \\
    &= \frac{x^0}{0!} + \frac{x^1}{1!} + \frac{x^2}{2!} + \frac{x^3}{3!} + \cdots
\end{align*}
\begin{align*}
    f'(x) &= 1\frac{x^0}{1!} + 2\frac{x^1}{2!} + 3\frac{x^2}{3!} + 4\frac{x^3}{4!} + \cdots \\
    &= \sum_{n=0}^\infty \frac{(n+1)x^n}{(n+1)!} \\
    &= \sum_{n=0}^\infty \frac{x^n}{n!}
\end{align*}
The derivative of this power series is just itself, which makes sense because this is the power series for $e^x$.

\section*{5.1.16}
The function in 5.1.14 is $f(x) = e^x$. First we figure out the series
\begin{align*}
     & \frac{x^0}{0!} - \frac{x^1}{1!} + \frac{x^2}{2!} - \frac{x^3}{3!} + \cdots \\
    =& \frac{x^0}{0!} + \frac{x^1}{1!} - 2\frac{x^1}{1!} + \frac{x^2}{2!} + \frac{x^3}{3!} - 2\frac{x^3}{3!} + \cdots \\
    =& \round{\frac{x^0}{0!} + \frac{x^1}{1!}  + \frac{x^2}{2!} + \frac{x^3}{3!} + \cdots} - 2 \round{\frac{x^1}{1!} + \frac{x^3}{3!} + \frac{x^5}{5!} + \cdots} \\
    =& \sum_{n=0}^\infty \frac{x^n}{n!} - 2\sum_{n=0}^\infty \frac{x^{2n+1}}{(2n+1)!} \\
    =& e^x - 2\sinh(x)
\end{align*}

Next we figure out the series
\begin{align*}
     & \frac{x^0}{0!} + \frac{x^2}{1!} + \frac{x^4}{2!} + \frac{x^6}{3!} + \cdots \\
    =& \frac{x^0}{0!} + \frac{x^1}{1!} - \frac{x^1}{1!} + \frac{x^2}{1!} + \frac{x^3}{3!} - \frac{x^3}{3!} + \frac{x^4}{2!} + \frac{x^5}{5!} - \frac{x^5}{5!} + \frac{x^6}{3!} + \cdots \\
    =& \round{\frac{x^0}{0!} + \frac{x^1}{1!} + \frac{x^2}{1!} + \frac{x^3}{3!} + \frac{x^4}{2!} + \cdots} - \round{\frac{x^1}{1!} + \frac{x^3}{3!} + \frac{x^5}{5!} + \cdots} \\
    =& \sum_{n=0}^\infty \frac{x^n}{n!} - \sum_{n=0}^\infty \frac{x^{2n+1}}{(2n+1)!} \\
    =& e^x - \sinh(x)
\end{align*}



\section*{5.2.12}

\subsection*{1}
Find the taylor series $g(x)$ of $f(x) = 1 + x + x^2$ at $c=-1$.

\begin{align*}
    f(x)    &= x^2 + x + 1 & f(-1)    &= 1  \\
    f'(x)   &= 2x + 1      & f'(-1)   &= -1 \\
    f''(x)  &= 2           & f''(-1)  &= 2  \\
    f'''(x) &= 0           & f'''(-1) &= 0  \\
\end{align*}
\begin{align*}
    g(x) &= \frac{1 (x-(-1))^0}{0!} + \frac{-1 (x-(-1))^1}{1!} + \frac{2 (x-(-1))^2}{2!} \\
    &= 1 - (x+1) + (x+1)^2 \\
    &= (x+1)^2 - x \\
    &= x^2 + 2x + 1 - x \\
    &= x^2 + x + 1
\end{align*}
The taylor series is the same as the original function.

\subsection*{3}
Find the taylor series $g(x)$ of $f(x) = e^x$ at $c=1$.

We know that $\forall n \in \N$, $f^{(n)}(x) = e^x$. Likewise, $\forall n \in \N$, $f^{(n)} (1) = e$. We then have the taylor series
\begin{align*}
    g(x) &= \sum_{n=0}^\infty \frac{e (x-1)^n}{n!} \\
\end{align*}

\subsection*{5}
Find the taylor series $g(x)$ of $f(x) = (x-1)^2 e^x$ at $c=1$.

$h(x) = (x-1)^2 = x^2 - 2x + 1$. We find the following derivatives:
\begin{align*}
    h(x)   &= x^2 - 2x + 1 & h(1)   &= 0 \\
    h'(x)  &= 2x - 2       & h'(1)  &= 0 \\
    h''(x) &= 2            & h''(1) &= 1
\end{align*}
The taylor series for $h(x)$ is
\begin{align*}
    h(x) &= \frac{0(x-1)^0}{0!} + \frac{0(x-1)^1}{1!} + \frac{2(x-1)^2}{2!} \\
         &= (x-1)^2
\end{align*}
We combine this with the taylor series for $e^x$ at $c=1$ from the previous problem and find
\begin{align*}
    g(x) &= \sum_{n=0}^\infty \frac{e(x-1)^{n+2}}{n!}
\end{align*}

\subsection*{6}
Find the taylor series $g(x)$ of $f(x) = \cos(2x)$ at $c=0$.

We notice that this is the same as the taylor series of $\cos(x)$ except that each derivative carries an extra multiple of 2. Therefore, we have
\begin{align*}
    g(x) = \sum_{n=0}^\infty \frac{(-1)^n (2x)^{2n}}{(2n)!}
\end{align*}

\subsection*{9}
Find the taylor series $g(x)$ of $f(x) = \ln(1+x)$ at $c=0$.

This is the same as the expansion of $\ln(x)$ at $c=1$ but with $x$ shifted.
\begin{align*}
    g(x) &= \sum_{n=0}^\infty \frac{(-1)^n x^{n+1}}{n+1}
\end{align*}



\section*{5.2.16}
Use known Taylor series to find the limits.

\subsection*{2}
\begin{align*}
     & \lim_{n\to0} \frac{\cos(x) - 1 - \frac{x^2}{2}}{x^4} \\
\end{align*}
I think that the problem was intended to be written slightly differently, and I will solve the modified version:
\begin{align*}
     & \lim_{x\to0} \frac{\cos(x) - 1 + \frac{x^2}{2}}{x^4} \\
    =& \lim_{x\to0} \frac{\sum_{n=0}^\infty \frac{(-1)^n x^{2n}}{(2n)!} - 1 + \frac{x^2}{2}}{x^4} \\
    =& \lim_{x\to0} \frac{\sum_{n=2}^\infty \frac{(-1)^n x^{2n}}{(2n)!}}{x^4} \\
    =& \lim_{x\to0} \sum_{n=2}^\infty \frac{(-1)^n x^{2n-4}}{(2n)!} \\
    =& \lim_{x\to0} \sum_{n=2}^\infty \frac{(-1)^n x^{2(n-2)}}{(2n)!} \\
    =& \lim_{x\to0} \sum_{n=0}^\infty \frac{(-1)^n x^{2n}}{(2n+4)!} \\
    =& \frac{1}{4!} = \frac{1}{24}
\end{align*}



\subsection*{3}
\begin{align*}
     & \lim_{x\to0} \frac{e^{x^2} - 1}{x^2} \\
    =& \lim_{x\to0} \frac{\sum_{n=0}^\infty \frac{\round{x^2}^n}{n!} - 1}{x^2} \\
    =& \lim_{x\to0} \frac{\sum_{n=1}^\infty \frac{x^{2n}}{n!}}{x^2} \\
    =& \lim_{x\to0} \sum_{n=1}^\infty \frac{x^{2n-2}}{n!} \\
    =& \lim_{x\to0} \sum_{n=1}^\infty \frac{x^{2(n-1)}}{n!} \\
    =& \lim_{x\to0} \sum_{n=0}^\infty \frac{x^{2n}}{(n+1)!} \\
    =& 1
\end{align*}
We see the last step because all terms are multiples of $x$ except for the first.



\subsection*{4}
\begin{align*}
     & \lim_{x\to0} \frac{\sin(x) - x}{x^3} \\
    =& \lim_{x\to0} \frac{\sum_{n=0}^\infty \frac{(-1)^n x^{2n+1}}{(2n+1)!} - x}{x^3} \\
    =& \lim_{x\to0} \frac{\sum_{n=1}^\infty \frac{(-1)^n x^{2n+1}}{(2n+1)!}}{x^3} \\
    =& \lim_{x\to0} \sum_{n=1}^\infty \frac{(-1)^n x^{2n-2}}{(2n+1)!} \\
    =& \lim_{x\to0} \sum_{n=0}^\infty \frac{(-1)^{n+1} x^{2n}}{(2n+3)!} \\
    =& \frac{-1}{3!} = -\frac{1}{6}
\end{align*}



\subsection*{6}
\begin{align*}
     & \lim_{x\to1} \frac{\sin(x-1) - x + 1}{(x-1)^3} \\
    =& \lim_{x\to0} \frac{\sin(x) - x}{x^3} \\
    =& -\frac{1}{6}
\end{align*}


\end{document}
